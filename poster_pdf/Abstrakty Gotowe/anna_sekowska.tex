\documentclass[12pt, a4paper]{article}

%polskie znaki

\usepackage{polski}
\usepackage[utf8]{inputenc}

%matematyczne znaki i strza?eczki
\usepackage{amsmath}
\usepackage{amssymb}
\usepackage{graphicx}
\usepackage{fancyhdr}
\pagestyle{fancy}
\fancypagestyle{title}{%
  %\setlength{\headheight}{22pt}%
  \fancyhf{}% No header/footer
  %\renewcommand{\headrulewidth}{0pt}% No header rule
  %\renewcommand{\footrulewidth}{0pt}% No footer rule
  \fancyfoot[R]{FIZYKA DLA MEDYKA 2016}% Page number in Centre of footer
  \fancyhead[R]{Sesja posterowa}

}%
%nowozdefiniowane symbole
\newcommand{\n}{\mathbb{N}}
\newcommand{\re}{\mathbb{R}}
\newcommand{\limm}{\lim \limits_{x\rightarrow x_0}}
\usepackage{graphicx}


\usepackage[left=1 cm, right=1 cm, top=2 cm, bottom=2 cm]{geometry}

\begin{document}

\title{Możliwość wykorzystania optycznych fantomów do kalibracji laserów dermatologicznych.}
\author{Anna Sękowska, Maciej S. Wróbel, Stanisław Galla, Adam Cenian \\Politechnika Gdańska\\ KN Biofoton}
\date {}
\maketitle
\thispagestyle{title}
Lasery znajdują szerokie zastosowanie w terapii chorób dermatologicznych. Jednakże zanim nowy laser zostanie dopuszczony do użytku, konieczne jest zbadanie jego parametrów i zdolności do interakcji z tkankami. Właśnie do tego potrzebne są fantomy optyczne, które dokładnie odzwierciedlają zdolność rozpraszania i absorpcji oraz właściwości termiczne skóry ludzkiej. Na potrzeby przeprowadzonych badań wytworzyliśmy zestaw fantomów o różnych parametrach optycznych i termicznych. Wykonaliśmy testy z wykorzystaniem lasera 975 nm, zmieniając jego ustawienia tj. moc, długość i ilość impulsów. Pomiaru czasowego i przestrzennego rozkładu temperatury na powierzchni fantomów i rzeczywistych tkanek dokonaliśmy za pomocą kamery termograficznej. Po porównaniu uzyskanych wyników byliśmy w stanie stwierdzić, że fantomy optyczne mogą być z powodzeniem stosowane do przedklinicznych testów oraz kalibracji laserów dermatologicznych.

\end{document}
