\documentclass[12pt, a4paper]{article}

%polskie znaki

\usepackage{polski}
\usepackage[utf8]{inputenc}

%matematyczne znaki i strza?eczki
\usepackage{amsmath}
\usepackage{amssymb}
\usepackage{graphicx}
\usepackage{fancyhdr}
\pagestyle{fancy}
\fancypagestyle{title}{%
  %\setlength{\headheight}{22pt}%
  \fancyhf{}% No header/footer
  %\renewcommand{\headrulewidth}{0pt}% No header rule
  %\renewcommand{\footrulewidth}{0pt}% No footer rule
  \fancyfoot[L]{FIZYKA DLA MEDYKA 2016}% Page number in Centre of footer
  \fancyhead[L]{Sesja posterowa}

}%
%nowozdefiniowane symbole
\newcommand{\n}{\mathbb{N}}
\newcommand{\re}{\mathbb{R}}
\newcommand{\limm}{\lim \limits_{x\rightarrow x_0}}
\usepackage{graphicx}


\usepackage[left=1 cm, right=1 cm, top=2 cm, bottom=2 cm]{geometry}

\begin{document}

\title{Inżynierskie wspomaganie przedoperacyjne przy wykorzystaniu metod szybkiego prototypowania.}
\author{inż. Magdalena Żukowska \\ Wydział Budowy Maszyn i Zarządzania Politechniki Poznańskiej}
\date {}
\maketitle
\thispagestyle{title}
Na przestrzeni ostatnich kilku lat obserwuje się dynamiczny rozwój metod szybkiego prototypowania, co ma bezpośredni wpływ na zmiany zachodzące w działach powiązanych z inżynierią biomedyczną oraz szeroko pojętą medycyną. Coraz częściej dokumentuje się powodzenie skomplikowanych operacji przeprowadzanych w oparciu o wykonany metodami przyrostowymi model medyczny, będący faktycznym odzwierciedleniem stanu organu pacjenta. Istotną kwestią we wspomaganiu przedoperacyjnym jest zaprezentowanie wizualizacji przestrzennej narządu zmienionego chorobowo. Dzięki wykorzystaniu obrazów TK/MRI, inżynier może wykonać segmentację, której efektem będzie obraz 3D. Działania te wsparte wykorzystaniem metod szybkiego prototypowania, pozwalają na sprawne wykonanie modelu odzwierciedlającego realny stan obszaru chorego. Uzyskana w ten sposób indywidualizacja, pozwala na bezpieczniejsze wykonanie operacji oraz znaczne skrócenie jej trwania, gdyż zarówno lekarz jak i pacjent mogą się lepiej do niej przygotować.

Celem pracy jest przedstawienie metodyki tworzenia modelu medycznego nerki ze zmianami ogniskowymi przy wykorzystaniu metod szybkiego prototypowania. Metodyka obejmuje wszelkie prace związane z segmentacją, obróbką modelu, generowaniem obiektu przestrzennego wraz z zapisem w formacie STL oraz doborem odpowiedniej metody szybkiego prototypowania. Dodatkowo praca ma charakter porównawczy. Skonfrontowane zostaną ze sobą dwie metody szybkiego prototypowania: FDM (Fused Deposition Modeling) oraz 3DP (3D Printing), przy użyciu których wykonano modele. 

\end{document}
