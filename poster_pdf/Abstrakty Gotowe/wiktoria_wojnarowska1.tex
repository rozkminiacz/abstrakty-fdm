\documentclass[12pt, a4paper]{article}

%polskie znaki

\usepackage{polski}
\usepackage[utf8]{inputenc}

%matematyczne znaki i strza?eczki
\usepackage{amsmath}
\usepackage{amssymb}
\usepackage{graphicx}
\usepackage{fancyhdr}
\pagestyle{fancy}
\fancypagestyle{title}{%
  %\setlength{\headheight}{22pt}%
  \fancyhf{}% No header/footer
  %\renewcommand{\headrulewidth}{0pt}% No header rule
  %\renewcommand{\footrulewidth}{0pt}% No footer rule
  \fancyfoot[L]{FIZYKA DLA MEDYKA 2016}% Page number in Centre of footer
  \fancyhead[L]{Sesja posterowa}

}%
%nowozdefiniowane symbole
\newcommand{\n}{\mathbb{N}}
\newcommand{\re}{\mathbb{R}}
\newcommand{\limm}{\lim \limits_{x\rightarrow x_0}}
\usepackage{graphicx}


\usepackage[left=1 cm, right=1 cm, top=2 cm, bottom=2 cm]{geometry}

\begin{document}

\title{Innowacyjna metoda w medycynie - 3D BIOPRINTING tkanek i~komórek}
\author{Wiktoria Wojnarowska, Anna Korbecka \\Politechnika Rzeszowska, WMiFS \\KN Foton}
\date {}
\maketitle
\thispagestyle{title}
Praca dotyczy zagadnień związanych z drukiem 3D, tzw. 3D bioprintingiem. Jest to innowacyjny sposób druku 3D tkanek miękkich. Jego historia sięga lat ’80. XX  wieku. Na posterze została opisana i ukazana na schemacie zasada działania biodruku. Również opisano przyrządy w niej używane. Ukazano także ostatnie osiągnięcia biodruku oraz nadzieje z nim związane. Zwrócono uwagę na wady i zalety. 

\end{document}
