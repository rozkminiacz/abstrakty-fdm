\documentclass[12pt, a4paper]{article}

%polskie znaki

\usepackage{polski}
\usepackage[utf8]{inputenc}

%matematyczne znaki i strza?eczki
\usepackage{amsmath}
\usepackage{amssymb}
\usepackage{graphicx}
\usepackage{fancyhdr}
\pagestyle{fancy}
\fancypagestyle{title}{%
  %\setlength{\headheight}{22pt}%
  \fancyhf{}% No header/footer
  %\renewcommand{\headrulewidth}{0pt}% No header rule
  %\renewcommand{\footrulewidth}{0pt}% No footer rule
  \fancyfoot[L]{FIZYKA DLA MEDYKA 2016}% Page number in Centre of footer
  \fancyhead[L]{Sesja posterowa}

}%
%nowozdefiniowane symbole
\newcommand{\n}{\mathbb{N}}
\newcommand{\re}{\mathbb{R}}
\newcommand{\limm}{\lim \limits_{x\rightarrow x_0}}
\usepackage{graphicx}


\usepackage[left=1 cm, right=1 cm, top=2 cm, bottom=2 cm]{geometry}

\begin{document}

\title{Prototypowanie endoprotezy stawu kolanowego z zastosowaniem technik wytwarzania przyrostowego.}
\author{inż. Anna Kwiatkowska \\ Wydział Budowy Maszyn i Zarządzania, Politechnika Poznańska}
\date {}
\maketitle
\thispagestyle{title}
W XXI wieku ciało człowieka nie stanowi już tajemnicy dla świata nauki. Dzisiejsza medycyna potrafi nie tylko diagnozować i leczyć choroby farmakologicznie lub poprzez resekcję, ale jest również w stanie sprostać wyzwaniu zastąpienia funkcji uszkodzonych narządów, m.in. poprzez implantację. Na szeroką skalę stosuje się implanty ortopedyczne (głównie endoprotezy stawów) i stomatologiczne.

Zdecydowana większość endoprotez produkowana jest na skalę masową, z wykorzystaniem kilku uniwersalnych modeli, występujących w kilku rozmiarach w celu ułatwienia dopasowania implantu do wymiarów kości konkretnego pacjenta. Znacznej części pacjentów można dobrać odpowiedni model i rozmiar implantu, jednak nadal istnieją przypadki, w których niemożliwe jest idealne dopasowanie endoprotezy. W konsekwencji może to prowadzić do powikłań pooperacyjnych, takich jak obluzowanie wszczepu lub metaloza. Dla takich pacjentów powinno się zastosować niestandardowe metody leczenia, np. użycie spersonalizowanego implantu. Indywidualne dopasowanie endoprotezy może pozwolić na zmniejszenie ryzyka wystąpienia niepożądanych skutków ubocznych, jak również przyczynić się do skrócenia czasu rekonwalescencji.

Wykorzystując dane z obrazowania medycznego (tomografia komputerowa), przeprowadzono operację segmentacji stawu kolanowego. Wykorzystując anatomiczną geometrię kości, zamodelowano elementy endoprotezy stawu kolanowego zindywidualizowanej w kierunku możliwie najlepszego dopasowania. Wykonano również wzorce kości umożliwiające wizualizację implantacji endoprotezy. Prototyp implantu oraz modele kości wytworzono za pomocą technik przyrostowych. Wykonano obróbkę wykańczającą powierzchni i przeprowadzono weryfikację stopnia dopasowania.

\end{document}
