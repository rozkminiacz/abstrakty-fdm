\documentclass[12pt, a4paper]{article}

%polskie znaki

\usepackage{polski}
\usepackage[utf8]{inputenc}

%matematyczne znaki i strza?eczki
\usepackage{amsmath}
\usepackage{amssymb}
\usepackage{graphicx}
\usepackage{fancyhdr}
\pagestyle{fancy}
\fancypagestyle{title}{%
  %\setlength{\headheight}{22pt}%
  \fancyhf{}% No header/footer
  %\renewcommand{\headrulewidth}{0pt}% No header rule
  %\renewcommand{\footrulewidth}{0pt}% No footer rule
  \fancyfoot[L]{FIZYKA DLA MEDYKA 2016}% Page number in Centre of footer
  \fancyhead[L]{Wykład specjalny}

}%
%nowozdefiniowane symbole
\newcommand{\n}{\mathbb{N}}
\newcommand{\re}{\mathbb{R}}
\newcommand{\limm}{\lim \limits_{x\rightarrow x_0}}
\usepackage{graphicx}


\usepackage[left=1 cm, right=1 cm, top=2 cm, bottom=2 cm]{geometry}

\begin{document}

\title{Medycyna nuklearna, cudowne dziecko fizyki medycznej}
\author{prof. dr hab. n. med. Leszek Królicki \\Zakład Medycyny Nuklearnej i Rezonansu Magnetycznego \\Akademia Medyczna w Warszawie}
\date {}
\maketitle
\thispagestyle{title}
\textbf{Piątek 1.04, 10:30 - 12:00}
\\

Medycyna Nuklearna jest samodzielną dziedziną medycyny. Metody z zakresu tej specjalności polegają na podaniu choremu odpowiedniego radiofarmaceutyku ( wybranej substancji chemicznej wyznakowanej radioizotopem) w celach diagnostycznych lub leczniczych. W zakresie metod diagnostycznych procedury radioizotopowe pozwalają na ocenę zaburzeń czynnościowych. Tym różnią się od innych metod obrazowych (TK, USG, MRI), które przedstawiają przede wszystkim zmiany strukturalne. W odniesieniu do leczenia – metody medycyny nuklearnej spełniają kryteria medycyny personalizowanej; u chorego wykonywane jest badanie, na podstawie którego oceniana jest farmakokinetyka radiofarmaceutyku, a następnie podawany jest ten sam – lub bardzo zbliżony – radiofarmaceutyk znakowany radioizotopem w dawkach leczniczych. Powstanie tej dziedziny wiąże się z czterema wydarzeniami. Pierwszym z nich było odkrycie w 1934 roku sztucznej promieniotwórczości przez Fryderyka i Irenę Joliot. Już w roku 1938 Lawrence zastosował radioaktywny fosfor (32P) do leczenia białaczki u 29 letniego studenta. Trzecim było zastosowanie przez Seidlina i wsp. w 1946 roku radioaktywnego jodu do leczenia raka tarczycy. Kolejnym wydarzeniem było uruchomienie w 1946 roku reaktora w OAKE RIDGE Laboratory, przeznaczonego do produkcji radioizotopów tylko do stosowania w medycynie. Otrzymywanie przydatnych dla medycyny radionuklidów z mieszaniny produktów rozszczepienia zostało opracowane w ramach „Manhattan Project” jeszcze podczas II Wojny Światowej, ale ze względu na konieczność zachowania w ścisłej tajemnicy wszelkich prac nad bronią jądrową nie uzyskano zgody na wykorzystanie istniejących możliwości. Dopiero rok po zakończeniu wojny, a dokładnie w numerze „Science” z dnia 14.06.1946 r ukazało się ogłoszenie o dostępności i możliwości zakupu w Oak Ridge National Laboratory dowolnych ilości sztucznych izotopów promieniotwórczych do celów medycznych. Tania metoda otrzymywania preparatów promieniotwórczych dla medycyny przestała być wreszcie tajemnicą militarną i mogła służyć człowiekowi. Pojęcie „medycyna nuklearna” zostało wprowadzone w roku 1952. Bardzo ważnym czynnikiem rozwoju medycyny nuklearnej był również postęp w zakresie aparatury pomiarowej. Pierwszym urządzeniem do obrazowania rozkładu promieniowania był scyntygraf skonstruowany przez B. Cassena w roku 1950. W1952 roku G. Brownell skonstruował pierwszy aparat do obrazowania pozytonowego, a w roku 1958 H. Anger przedstawił gamma-kamerę. W 1962 roku został przedstawiony pierwszy wielorzędowy aparat do badań PET. Następnym krokiem milowym okazały się systemy hybrydowe: SPECT-CT, PET-CT, PET-MRI. Systemy te sprawiają, że obecne badania obrazowe łączą w sobie osiągnięcia naukowców różnych dziedzin. Dalsza droga rozwoju medycyny nuklearnej jest już w Waszych rękach. Jestem przekonany, że nowe – Państwa - koncepcje spowodują kolejną rewolucję w fizyce medycznej.

\end{document}
