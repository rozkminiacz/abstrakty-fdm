\documentclass[12pt, a4paper]{article}

%polskie znaki

\usepackage{polski}
\usepackage[utf8]{inputenc}

%matematyczne znaki i strza?eczki
\usepackage{amsmath}
\usepackage{amssymb}
\usepackage{graphicx}
\usepackage{fancyhdr}
\pagestyle{fancy}
\fancypagestyle{title}{%
  %\setlength{\headheight}{22pt}%
  \fancyhf{}% No header/footer
  %\renewcommand{\headrulewidth}{0pt}% No header rule
  %\renewcommand{\footrulewidth}{0pt}% No footer rule
  \fancyfoot[R]{FIZYKA DLA MEDYKA 2016}% Page number in Centre of footer
  \fancyhead[R]{Wykład specjalny}

}%
%nowozdefiniowane symbole
\newcommand{\n}{\mathbb{N}}
\newcommand{\re}{\mathbb{R}}
\newcommand{\limm}{\lim \limits_{x\rightarrow x_0}}
\usepackage{graphicx}


\usepackage[left=1 cm, right=1 cm, top=2 cm, bottom=2 cm]{geometry}

\begin{document}

\title{RF resonator systems in MRI: developments and applications}
\author{M.Sc. Jorge Chacon-Caldera \\Medizinische Fakultät Mannheim der Universität Heidelberg, \\Universitätsklinikum Mannheim}
\date {}
\maketitle
\thispagestyle{title}
\textbf{Sobota 1.04, 10:30 - 11:30}
\\

Magnetic Resonance Imaging (MRI) is a medical imaging modality that offers the possibility to investigate anatomical and physiological features of the human body. MRI antennas resonant in radio-frequency ranges (RF resonators) are used for the transmission and reception of signals. This talk will focus on the developments and applications of such RF resonators. The topic of resonators will then be extended to arrays and purpose-built resonators for specific applications. Finally, state-of-the-art RF resonator systems, current research approaches and novel developments will be presented.

\end{document}
