\documentclass[12pt, a4paper]{article}

%polskie znaki

\usepackage{polski}
\usepackage[utf8]{inputenc}

%matematyczne znaki i strza?eczki
\usepackage{amsmath}
\usepackage{amssymb}
\usepackage{graphicx}
\usepackage{fancyhdr}
\pagestyle{fancy}
\fancypagestyle{title}{%
  %\setlength{\headheight}{22pt}%
  \fancyhf{}% No header/footer
  %\renewcommand{\headrulewidth}{0pt}% No header rule
  %\renewcommand{\footrulewidth}{0pt}% No footer rule
  \fancyfoot[R]{FIZYKA DLA MEDYKA 2016}% Page number in Centre of footer
  \fancyhead[R]{Sesja posterowa}

}%
%nowozdefiniowane symbole
\newcommand{\n}{\mathbb{N}}
\newcommand{\re}{\mathbb{R}}
\newcommand{\limm}{\lim \limits_{x\rightarrow x_0}}
\usepackage{graphicx}


\usepackage[left=1 cm, right=1 cm, top=2 cm, bottom=2 cm]{geometry}

\begin{document}

\title{Wpływ ultradźwięków na materiały}
\author{Niemiec Aleksandra, Śmierciak Marta, Zagrobelna Magdalena \\Politechnika Rzeszowska im. Ignacego Łukasiewicza\\ KN Foton}
\date {}
\maketitle
\thispagestyle{title}
Praca przedstawia fizyczne aspekty zastosowania ultradźwięków w zabiegu termoablacji ultradźwiękowej. Opisane zostały mechaniczne, termiczne i fizykochemiczne skutki oddziaływania ultradźwięków na materiały. Określono parametry właściwości tworzyw, mające znaczący wpływ na rozwój omawianej aplikacji medycznej. Praca daje podstawy fizyczne realizacji manipulatorów chroniących narządy wrażliwe na działanie ultradźwięków.

\end{document}
