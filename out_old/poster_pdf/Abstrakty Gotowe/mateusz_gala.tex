\documentclass[12pt, a4paper]{article}

%polskie znaki

\usepackage{polski}
\usepackage[utf8]{inputenc}

%matematyczne znaki i strza?eczki
\usepackage{amsmath}
\usepackage{amssymb}
\usepackage{graphicx}
\usepackage{fancyhdr}
\pagestyle{fancy}
\fancypagestyle{title}{%
  %\setlength{\headheight}{22pt}%
  \fancyhf{}% No header/footer
  %\renewcommand{\headrulewidth}{0pt}% No header rule
  %\renewcommand{\footrulewidth}{0pt}% No footer rule
  \fancyfoot[R]{FIZYKA DLA MEDYKA 2016}% Page number in Centre of footer
  \fancyhead[R]{Sesja posterowa}

}%
%nowozdefiniowane symbole
\newcommand{\n}{\mathbb{N}}
\newcommand{\re}{\mathbb{R}}
\newcommand{\limm}{\lim \limits_{x\rightarrow x_0}}
\usepackage{graphicx}


\usepackage[left=1 cm, right=1 cm, top=2 cm, bottom=2 cm]{geometry}

\begin{document}

\title{Badanie zmian pierwiastkowych w hipokampach szczurów jako narzędzie diagnostyki epilepsyjnej.}
\author{Mateusz Gala \\   
Akademia Górniczo-Hutnicza im. Stanisława Staszica w Krakowie \\
Wydział Fizyki i Informatyki Stosowanej
\\
SKNFM Kerma}
\date {}
\maketitle
\thispagestyle{title}
Badanie zmian pierwiastkowych w hipokampach szczurów jako narzędzie diagnostyki epilepsyjnej
Projekt ma na celu przebadanie 3 grup zwierząt, które odpowiadają różnemu wiekowi biologicznemu u człowieka. Mózgi szczurów zostały poddane analizie spektrometrycznej oraz mikroskopowej. Istotą pracy jest wykazanie różnic w składzie pierwiastkowym hipokampów pochodzących od osobników z różnych grup wiekowych.


\end{document}
