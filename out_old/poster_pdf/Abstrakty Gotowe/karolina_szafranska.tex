\documentclass[12pt, a4paper]{article}

%polskie znaki

\usepackage{polski}
\usepackage[utf8]{inputenc}

%matematyczne znaki i strza?eczki
\usepackage{amsmath}
\usepackage{amssymb}
\usepackage{graphicx}
\usepackage{fancyhdr}
\pagestyle{fancy}
\fancypagestyle{title}{%
  %\setlength{\headheight}{22pt}%
  \fancyhf{}% No header/footer
  %\renewcommand{\headrulewidth}{0pt}% No header rule
  %\renewcommand{\footrulewidth}{0pt}% No footer rule
  \fancyfoot[L]{FIZYKA DLA MEDYKA 2016}% Page number in Centre of footer
  \fancyhead[L]{Sesja posterowa}

}%
%nowozdefiniowane symbole
\newcommand{\n}{\mathbb{N}}
\newcommand{\re}{\mathbb{R}}
\newcommand{\limm}{\lim \limits_{x\rightarrow x_0}}
\usepackage{graphicx}


\usepackage[left=1 cm, right=1 cm, top=2 cm, bottom=2 cm]{geometry}

\begin{document}

\title{Badanie morfologii komórek śródbłonka wątrobowego przy użyciu mikroskopii sił atomowych oraz mikroskopii fluorescencyjnej.}
\author{Karolina Szafrańska \\Uniwersytet Jagielloński}
\date {}
\maketitle
\thispagestyle{title}
Komórki śródbłonka wątrobowego (LSEC – ang. Liver Sinusoidal Endothelial Cells) posiadając charakterystyczne struktury – fenestracje, będące nie posiadającymi membrany otworami w błonie komórkowej umożliwiającymi przenikanie substancji pomiędzy światłem naczyń a przestrzenią Dissego. Ich częściowy zanik lub całkowita nieobecność świadczyć może o patologicznych stanach narządu. Ze względu na niewielkie rozmiary pojedynczych fenestracji, wynoszące 80-200 nm, ich obrazowanie nie jest możliwe przy wykorzystaniu tradycyjnych metod optycznych. Dzięki zastosowaniu mikroskopii sił atomowych (AFM) możliwe staje się badanie tych niezwykle delikatnych struktur w warunkach zbliżonych do fizjologicznych. Mysie, izolowane komórki śródbłonka wątrobowego zobrazowano wykorzystując różne tryby pracy oraz określono wpływ rodzaju i stężenia utrwalacza na możliwości badania fenestracji. Uzyskane w ten sposób wyniki mogą służyć do późniejszej oceny kondycji śródbłonka wątrobowego w stanach patologicznych takich jak niealkoholowe stłuszczenie wątroby.

\end{document}
