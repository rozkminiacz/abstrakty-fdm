\documentclass[12pt, a4paper]{article}

%polskie znaki

\usepackage{polski}
\usepackage[utf8]{inputenc}

%matematyczne znaki i strza?eczki
\usepackage{amsmath}
\usepackage{amssymb}
\usepackage{graphicx}
\usepackage{fancyhdr}
\pagestyle{fancy}
\fancypagestyle{title}{%
  %\setlength{\headheight}{22pt}%
  \fancyhf{}% No header/footer
  %\renewcommand{\headrulewidth}{0pt}% No header rule
  %\renewcommand{\footrulewidth}{0pt}% No footer rule
  \fancyfoot[L]{FIZYKA DLA MEDYKA 2016}% Page number in Centre of footer
  \fancyhead[L]{Sesja posterowa}

}%
%nowozdefiniowane symbole
\newcommand{\n}{\mathbb{N}}
\newcommand{\re}{\mathbb{R}}
\newcommand{\limm}{\lim \limits_{x\rightarrow x_0}}
\usepackage{graphicx}


\usepackage[left=1 cm, right=1 cm, top=2 cm, bottom=2 cm]{geometry}

\begin{document}

\title{Wykorzystanie tympanometru do oceny stanu ucha środkowego u pacjentki z niedosłuchem.}
\author{Paulina Chalińska, Maciej Paliwoda \\ Wydział Inżynierii Produkcji i Technologii Materiałów, Politechnika Częstochowska}
\date {}
\maketitle
\thispagestyle{title}
     W pracy przedstawiono wyniki badań tympanometrycznych czterolatki, u której rodzice zauważyli postępujący ubytek słuchu. Na początku wykluczono przyczyny, które mogły by powodować niedosłuch typu przewodzeniowego na poziomie przewodu słuchowego zewnętrznego. Następnie przystąpiono do zbadania ucha środkowego dziewczynki, w celu sprawdzenia czy nie zalega tam płyn powodujący wysiękowe zapalenie ucha środkowego. Jest to jedna z najczęstszych chorób układu słuchu, podczas której za błoną bębenkową gromadzi się i zalega tzw. wysięk. Zbyt późne wykrycie tej dolegliwości może negatywnie rzutować na zdrowie pacjenta, a w konsekwencji powodować całkowitą utratę słuchu. Na podstawie otrzymanych tympanogramów stwierdzono, że w uchu środkowym zalega płyn wysiękowy. Świadczy o tym płaska krzywa bez wyraźnego maksimum (typ B) dla obojga uszu. Po przeanalizowaniu wyników badania, podjęto leczenie farmakologiczne, które trwało sześć miesięcy. W obliczu nieskuteczności stosowanej terapii farmakologicznej zdecydowano się na zabieg chirurgiczny polegający na perforacji błony bębenkowej i zastosowaniu drenów wentylacyjnych. Po roku przeprowadzono kolejne badania które potwierdziły skuteczność zabiegu i prawidłowe funkcjonowanie ucha środkowego pacjentki.

\end{document}
