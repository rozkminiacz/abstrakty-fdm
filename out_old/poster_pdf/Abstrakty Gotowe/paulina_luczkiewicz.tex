\documentclass[12pt, a4paper]{article}

%polskie znaki

\usepackage{polski}
\usepackage[utf8]{inputenc}

%matematyczne znaki i strza?eczki
\usepackage{amsmath}
\usepackage{amssymb}
\usepackage{graphicx}
\usepackage{fancyhdr}
\pagestyle{fancy}
\fancypagestyle{title}{%
  %\setlength{\headheight}{22pt}%
  \fancyhf{}% No header/footer
  %\renewcommand{\headrulewidth}{0pt}% No header rule
  %\renewcommand{\footrulewidth}{0pt}% No footer rule
  \fancyfoot[R]{FIZYKA DLA MEDYKA 2016}% Page number in Centre of footer
  \fancyhead[R]{Sesja posterowa}

}%
%nowozdefiniowane symbole
\newcommand{\n}{\mathbb{N}}
\newcommand{\re}{\mathbb{R}}
\newcommand{\limm}{\lim \limits_{x\rightarrow x_0}}
\usepackage{graphicx}


\usepackage[left=1 cm, right=1 cm, top=2 cm, bottom=2 cm]{geometry}

\begin{document}

\title{Modyfikacja interferometrycznego układu do obrazowania efektów oddziaływania światła laserowego na opatrunki hydrożelowe.}
\author{inż. Paulina Łuczkiewicz, prof. dr hab. Ewa Stachowska, dr Frans Meijer \\ Wydział Budowy Maszyn i Zarządzania Politechniki Poznańskiej}
\date {}
\maketitle
\thispagestyle{title}
Laseroterapia jest stosunkowo nową i prężnie rozwijającą się dziedziną nauki. Jej początki sięgają drugiej połowy XX wieku. Wraz z upływem czasu zakres wykorzystywania laserów w medycynie stale się powiększa. Na szczególna uwagę zasługuje wykorzystanie laserów terapeutycznych w dermatologii i medycynie estetycznej. Dzięki zabiegom laserowym możliwe stało się m.in. rozjaśnianie znamion i przebarwień bez interwencji chirurgicznej, zamykanie naczynek oraz trwałe usuwanie owłosienia, tatuaży i brodawek. Oddziaływanie lasera na tkankę skutkuje nieprzyjemnym uczuciem spowodowanym absorpcją promieniowania przez skórę, dlatego podczas zabiegów dermatologicznych coraz częściej wykorzystuje się również opatrunki hydrożelowe. Dzięki ich zastosowaniu duża grupa pacjentów przyznaje, że ból jest zdecydowanie mniejszy. Opatrunki hydrożelowe mogą być wykorzystane również po zabiegu do zabezpieczenia miejsca uwrażliwionego przez działanie lasera.

W pracy przeprojektowano i przebudowano układ optyczny z wykorzystaniem interferometru Macha-Zehndera i lasera He-Ne do badania właściwości opatrunków hydrożelowych (Kikgel HydroAid). Korzystając z tego układu zbadano wpływ promieniowania laserów terapeutycznych:  Fotona Dualis SP (laser Nd:YAG i Er:YAG) oraz QX Max (laser Nd:YAG) na opatrunki hydrożelowe, w zależności od parametrów wiązek terapeutycznych: długość fali, fluencji, częstotliwości i czasu trwania impulsu laserowego. 

\end{document}
