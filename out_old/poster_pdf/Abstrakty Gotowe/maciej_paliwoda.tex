\documentclass[12pt, a4paper]{article}

%polskie znaki

\usepackage{polski}
\usepackage[utf8]{inputenc}

%matematyczne znaki i strza?eczki
\usepackage{amsmath}
\usepackage{amssymb}
\usepackage{graphicx}
\usepackage{fancyhdr}
\pagestyle{fancy}
\fancypagestyle{title}{%
  %\setlength{\headheight}{22pt}%
  \fancyhf{}% No header/footer
  %\renewcommand{\headrulewidth}{0pt}% No header rule
  %\renewcommand{\footrulewidth}{0pt}% No footer rule
  \fancyfoot[R]{FIZYKA DLA MEDYKA 2016}% Page number in Centre of footer
  \fancyhead[R]{Sesja posterowa}

}%
%nowozdefiniowane symbole
\newcommand{\n}{\mathbb{N}}
\newcommand{\re}{\mathbb{R}}
\newcommand{\limm}{\lim \limits_{x\rightarrow x_0}}
\usepackage{graphicx}


\usepackage[left=1 cm, right=1 cm, top=2 cm, bottom=2 cm]{geometry}

\begin{document}

\title{Ustalenie progu słyszalności w zależności od czasu ekspozycji na hałas.}
\author{Maciej Paliwoda, Paulina Chalińska \\ Wydział Inżynierii Produkcji i Technologii Materiałów}
\date {}
\maketitle
\thispagestyle{title}
W pracy zamieszczone zostały wyniki badań progowej audiometrii tonalnej dla przewodnictwa powietrznego, wykonanej na grupie 15 osób w przedziale wiekowym 28-55 lat – pracujących w szkole podstawowej. Budynek szkoły znajduje się przy ruchliwej drodze krajowej, część kadry nauczycielskiej skarży się na problemy słuchowe wynikające z długotrwałego przebywania w środowisku zanieczyszczonym hałasem. Pomiary słuchu odbywały się dwuetapowo. Pierwszą część badań przeprowadzono przed rozpoczęciem zajęć lekcyjnych. Druga część pomiarów wykonana została pod koniec dnia pracy. Następnie porównano otrzymane rezultaty.

\end{document}
