\documentclass[12pt, a4paper]{article}

%polskie znaki

\usepackage{polski}
\usepackage[utf8]{inputenc}

%matematyczne znaki i strza?eczki
\usepackage{amsmath}
\usepackage{amssymb}
\usepackage{graphicx}
\usepackage{fancyhdr}
\pagestyle{fancy}
\fancypagestyle{title}{%
  %\setlength{\headheight}{22pt}%
  \fancyhf{}% No header/footer
  %\renewcommand{\headrulewidth}{0pt}% No header rule
  %\renewcommand{\footrulewidth}{0pt}% No footer rule
  \fancyfoot[L]{FIZYKA DLA MEDYKA 2016}% Page number in Centre of footer
  \fancyhead[L]{Sesja referatowa}

}%

%nowozdefiniowane symbole
\newcommand{\n}{\mathbb{N}}
\newcommand{\re}{\mathbb{R}}
\newcommand{\limm}{\lim \limits_{x\rightarrow x_0}}
\usepackage{graphicx}


\usepackage[left=1 cm, right=1 cm, top=2 cm, bottom=2 cm]{geometry}

\begin{document}

\title{Wybrane profile źródeł zanieczyszczeń pyłowych powietrza}
\author{Agata Skrzypek, Żaneta Czech \\Akademia Górniczo-Hutnicza im. Stanisława Staszica w Krakowie\\Wydział Fizyki i Informatyki Stosowanej\\ SKNFM Kerma}
\date {}
\maketitle
\thispagestyle{title}
W południowej Polsce wciąż obserwuje się wysoki poziom zanieczyszczeñ pyłowych powietrza. Zanieczyszczenia te ujemnie wpływają na zdrowie ludzi powodując liczne choroby. Poznanie źródeł zanieczyszczeñ pozwoli ograniczyć emisję. Profile zanieczyszczeñ pyłowych powietrza stanowią dobrą i ważną bazę danych pomocną przy szacowaniu źródeł zanieczyszczeñ metodami statystycznymi.
Celem prezentowanej pracy jest poznanie składu pierwiastkowego wybranych profili źródeł zanieczyszczeñ pyłowych powietrza oraz ustalenie relacji pomiędzy stężeniami wybranych pierwiastków.

     	Do wyznaczenia stężeń pierwiastków zastosowano metodę fluorescencji rentgenowskiej. Zaletą metody jest fakt, że próbka nie ulega zniszczeniu podczas analizy oraz jednocześnie otrzymuje się widmo zawierające cały zestaw pierwiastków. Analizowane były próbki pochodzące ze spalania drewna (buk), z silników Diesla, benzynowych oraz z katalizatorem, a także próbki pobrane podczas emisji z komina elektrociepłowni. Próbki te pochodzą od dr inż. K. Styszko, Wydział Paliw i Energii, Akademii Górniczo Hutniczej w Krakowie. Podczas spalania drewna emitowane są następujące pierwiastki: Cl, K, Ca oraz śladowe ilości Mn, Fe, Zn. Skład pyłów emitowanych z silników Diesla jest podobny do tych emitowanych z silnika benzynowego. Pyły te zawierają du¿e stężenia Ba, Zn, Ca, K, Fe, Sr oraz śladowe ilości Co, Cu, Br. Natomiast pyły emitowane z elektrociepłowni zawierają całą gamę pierwiastków takich jak K, Ca, Ti, Fe, Ba w ilościach znaczących oraz śladowe ilości Mn, Ni, Cu, Zn, As, Se, Br, Rb, Sr.

     	Otrzymana baza danych zawierająca wybrane profile stanowi doskonały materiał do wykorzystania w badaniach określających rodzaje źródeł i ich udział w całkowitej masie zanieczyszczeñ pyłowych powietrza. Prowadzone badania mogą przyczynić się do ograniczenia emisji w regionie.

\end{document}
