\documentclass[12pt, a4paper]{article}

%polskie znaki

\usepackage{polski}
\usepackage[utf8]{inputenc}

%matematyczne znaki i strza?eczki
\usepackage{amsmath}
\usepackage{amssymb}
\usepackage{graphicx}
\usepackage{fancyhdr}
\pagestyle{fancy}
\fancypagestyle{title}{%
  %\setlength{\headheight}{22pt}%
  \fancyhf{}% No header/footer
  %\renewcommand{\headrulewidth}{0pt}% No header rule
  %\renewcommand{\footrulewidth}{0pt}% No footer rule
  \fancyfoot[R]{FIZYKA DLA MEDYKA 2016}% Page number in Centre of footer
  \fancyhead[R]{Sesja referatowa}

}%

%nowozdefiniowane symbole
\newcommand{\n}{\mathbb{N}}
\newcommand{\re}{\mathbb{R}}
\newcommand{\limm}{\lim \limits_{x\rightarrow x_0}}
\usepackage{graphicx}


\usepackage[left=1 cm, right=1 cm, top=2 cm, bottom=2 cm]{geometry}

\begin{document}

\title{Możliwości zastosowania sygnałów biologicznych (EMG) w systemach układów sterowania}
\author{Mikołaj Kegler \\Politechnika Warszawska \\Koło Naukowe Aparatury Biomedycznej}
\date {}
\maketitle
\thispagestyle{title}
Referat ma na celu przedstawienie słuchaczom teorii biologicznego sprzężenia zwrotnego z wykorzystaniem potencjału elektrycznego aktywności mięśni (EMG). Układy takie stosowane są do realizacji zadań sterowania urządzeniami. W treści referatu przedstawione są dotychczas opracowane rozwiązania na podstawie źródeł literaturowych oraz efektów prac badawczych Politechniki Warszawskiej. W skład prezentacji będzie wchodzić przegląd urządzeń służących do detekcji oraz przetwarzania sygnału EMG, jak i obiektów sterowania.

\end{document}
