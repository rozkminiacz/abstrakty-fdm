\documentclass[12pt, a4paper]{article}

%polskie znaki

\usepackage{polski}
\usepackage[utf8]{inputenc}

%matematyczne znaki i strza?eczki
\usepackage{amsmath}
\usepackage{amssymb}
\usepackage{graphicx}
\usepackage{fancyhdr}
\pagestyle{fancy}
\fancypagestyle{title}{%
  %\setlength{\headheight}{22pt}%
  \fancyhf{}% No header/footer
  %\renewcommand{\headrulewidth}{0pt}% No header rule
  %\renewcommand{\footrulewidth}{0pt}% No footer rule
  \fancyfoot[L]{FIZYKA DLA MEDYKA 2016}% Page number in Centre of footer
  \fancyhead[L]{Sesja referatowa}

}%

%nowozdefiniowane symbole
\newcommand{\n}{\mathbb{N}}
\newcommand{\re}{\mathbb{R}}
\newcommand{\limm}{\lim \limits_{x\rightarrow x_0}}
\usepackage{graphicx}


\usepackage[left=1 cm, right=1 cm, top=2 cm, bottom=2 cm]{geometry}

\begin{document}

\title{Bioniczne wzmacniacze ludzkiego ciała - egzoszkielety}
\author{Kamil Sandecki, Agnieszka Wilczek, Patryk Zagrodnik \\Politechnika Rzeszowska im. Ignacego Łukasiewicza}
\date {}
\maketitle
\thispagestyle{title}
Referat ma charakter przeglądowy. Jego celem jest przybliżenie słuchaczom podstawowych zagadnień związanych z egzoszkieletami czyli zewnętrznymi wzmacniaczami ludzkiego ciała. Przedstawione zostanie ich wykorzystanie w wielu dziedzinach takich jak rehabilitacja, militaria, przemysł oraz kosmonautyka. Zaprezentowane zostaną wiodące marki wykorzystywane i doskonalone przez wiele firm i państw (np. HAL, X1). Podczas odczytu słuchacze zostaną zapoznani z krótkim rysem historycznym oraz podstawowymi rozwiązaniami technicznymi stosowanymi w egzoszkietach umożliwiającymi im funkcjonowanie. Ponadto pokrótce przedstawione zostaną systemy wykorzystywane do odczytywania ruchów człowieka, sterowania, zasilania. W referacie poruszona została również kwestia materiałów wykorzystywanych do tworzenia protez zewnętrznych.

\end{document}
