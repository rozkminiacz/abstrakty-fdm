\documentclass[12pt, a4paper]{article}

%polskie znaki

\usepackage{polski}
\usepackage[utf8]{inputenc}

%matematyczne znaki i strza?eczki
\usepackage{amsmath}
\usepackage{amssymb}
\usepackage{graphicx}
\usepackage{fancyhdr}
\pagestyle{fancy}
\fancypagestyle{title}{%
  %\setlength{\headheight}{22pt}%
  \fancyhf{}% No header/footer
  %\renewcommand{\headrulewidth}{0pt}% No header rule
  %\renewcommand{\footrulewidth}{0pt}% No footer rule
  \fancyfoot[R]{FIZYKA DLA MEDYKA 2016}% Page number in Centre of footer
  \fancyhead[R]{Sesja referatowa}

}%
%nowozdefiniowane symbole
\newcommand{\n}{\mathbb{N}}
\newcommand{\re}{\mathbb{R}}
\newcommand{\limm}{\lim \limits_{x\rightarrow x_0}}
\usepackage{graphicx}


\usepackage[left=1 cm, right=1 cm, top=2 cm, bottom=2 cm]{geometry}
\pagestyle{empty}
\begin{document}

\title{Skuteczność stosowania neuromonitoringu jako metody zapobiegania porażeniu nerwu VII podczas zabiegów na śliniance przyusznej}
\author{Alicja Strzałka, Aleksandra Szotakowska \\Uniwersytet Medyczny w Łodzi}
\date {}
\maketitle
\thispagestyle{title}
Śródoperacyjny monitoring nerwu VII jest metodą wykorzystywaną m.in. podczas operacji operacjach ślinianki przyusznej, kąta mostowo-móżdżkowego, czy założenia implantu ślimakowego. Umożliwia identyfikację, prześledzenie przebiegu oraz ocenę czynności nerwu, dlatego uznaje się, że dzięki jego zastosowaniu maleje ryzyko powikłań pod postacią porażenia lub niedowładu. 
Celem pracy jest próba określenia realnego znaczenia śródoperacyjnego monitoringu nerwu twarzowego dla zachowania funkcji nerwu twarzowego po operacjach ślinianek przyusznych na podstawie przeglądu piśmiennictwa.
Poglądy autorów na realne znaczenie neuromonitoringu różnią się: jedne badania dowodzą, że nie ma to związku z lepszym rokowaniem (Grosheva M), inne udowadniają zbawienny wręcz wpływ stosowania tej techniki na częstość występowania porażenia nerwu VII (Wolf SR, López M, Terrel JM) lub wskazują zupełnie inne niż zastosowanie tej metody czynniki, od których zależy powodzenie operacji (Bittar RF, Lowry TR).
Mimo, że realna wartość śródoperacyjnego neuromonitoringu pozostaje kwestią dyskusyjną, przesłanki do jego stosowania oparte na podstawach neurofizjologii klinicznej są ciągle najistotniejszym i nieulegającym wątpliwości powodem, aby zamiast zaniechać jego używania, rozwijać i doskonalić tę metodę.

\end{document}
