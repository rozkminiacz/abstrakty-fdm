\documentclass[12pt, a4paper]{article}

%polskie znaki

\usepackage{polski}
\usepackage[utf8]{inputenc}

%matematyczne znaki i strza?eczki
\usepackage{amsmath}
\usepackage{amssymb}
\usepackage{graphicx}
\usepackage{fancyhdr}
\pagestyle{fancy}
\fancypagestyle{title}{%
  %\setlength{\headheight}{22pt}%
  \fancyhf{}% No header/footer
  %\renewcommand{\headrulewidth}{0pt}% No header rule
  %\renewcommand{\footrulewidth}{0pt}% No footer rule
  \fancyfoot[L]{FIZYKA DLA MEDYKA 2016}% Page number in Centre of footer
  \fancyhead[L]{Wykład specjalny}

}%
%nowozdefiniowane symbole
\newcommand{\n}{\mathbb{N}}
\newcommand{\re}{\mathbb{R}}
\newcommand{\limm}{\lim \limits_{x\rightarrow x_0}}
\usepackage{graphicx}


\usepackage[left=1 cm, right=1 cm, top=2 cm, bottom=2 cm]{geometry}

\begin{document}

\title{Perspektywy budowy zaawansowanej protezy dla niewidzących}
\author{dr inż. Paweł Hottowy \\Akademia Górniczo-Hutnicza im. Stanisława Staszica w Krakowie \\Wydział Fizyki i Informatyki Stosowanej}
\date {}
\maketitle
\thispagestyle{title}
\textbf{Sobota 2.04, 15:00 - 16:00}
\\

Siatkówka oka – cienka warstwa tkanki nerwowej wyściełająca dno oka – jest kluczowym elementem układu wzrokowego. Widziany przez oko obraz jest w niej przetwarzany kolejno przez komórki światłoczułe, wyspecjalizowane interneurony i w końcu przez komórki zwojowe, które wysyłają do mózgu informację wizualną zakodowaną w postaci sekwencji impulsów nerwowych. Choroby siatkówki, takie jak retinopatia barwnikowa czy zwyrodnienie plamki żółtej, prowadzą do stopniowego zaniku widzenia wskutek degeneracji komórek światłoczułych. Jednak komórki zwojowe pozostają żywe i nawet w zaawansowanym stadium choroby zachowują zdolność do generacji impulsów nerwowych i wysyłania ich do mózgu. Otwiera to możliwość zbudowania elektronicznej protezy siatkówki dla niewidomych, w której obraz z miniaturowej kamery przetwarzany jest na serie impulsów elektrycznych pobudzających komórki zwojowe siatkówki i w ten sposób dostarczających do mózgu pacjenta informację wizualną. Pierwsze urządzenia bazujące na tej idei pojawiły się już na rynku. Ich zasadniczym ograniczeniem jest jednak niska rozdzielczość przestrzenna stymulacji elektrycznej – duże elektrody stymulują jednocześnie wiele neuronów, przez co przesyłane do mózgu informacje wizualne są mało precyzyjne, a w pewnych sytuacjach niespójne. W efekcie jakość oferowanego przez współczesne protezy sztucznego widzenia jest daleka od doskonałości. Punktem wyjścia do wykładu będzie pytanie: czy możliwe jest stworzenie inteligentnego implantu, zdolnego dostarczyć do mózgu niewidomego pacjenta informację identyczną do tej, jaka powstaje w zdrowej siatkówce przetwarzającej złożoną informację wizualną? Postaram się odpowiedzieć z punktu widzenia współczesnej wiedzy na temat działania siatkówki, rozwoju technologii mikroelektronicznych i nanofabrykacji, oraz ostatnich wyników laboratoryjnych dotyczących precyzyjnej stymulacji elektrycznej komórek nerwowych.

\end{document}
