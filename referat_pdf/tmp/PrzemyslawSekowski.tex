\documentclass[12pt, a4paper]{article}

%polskie znaki

\usepackage{polski}
\usepackage[utf8]{inputenc}

%matematyczne znaki i strza?eczki
\usepackage{amsmath}
\usepackage{amssymb}
\usepackage{graphicx}
\usepackage{fancyhdr}
\pagestyle{fancy}
\fancypagestyle{title}{%
  %\setlength{\headheight}{22pt}%
  \fancyhf{}% No header/footer
  %\renewcommand{\headrulewidth}{0pt}% No header rule
  %\renewcommand{\footrulewidth}{0pt}% No footer rule
  \fancyfoot[L]{FIZYKA DLA MEDYKA 2016}% Page number in Centre of footer
  \fancyhead[L]{Sesja referatowa}

}%

%nowozdefiniowane symbole
\newcommand{\n}{\mathbb{N}}
\newcommand{\re}{\mathbb{R}}
\newcommand{\limm}{\lim \limits_{x\rightarrow x_0}}
\usepackage{graphicx}


\usepackage[left=1 cm, right=1 cm, top=2 cm, bottom=2 cm]{geometry}

\begin{document}

\title{Badanie wpływu wartości średniego potencjału jonizacyjnego ośrodka na zasięg jonów stosowanych w hadronoterapii.}
\author{Przemysław Sękowski \\ Uniwersytet Warszawski}
\date {}
\maketitle
\thispagestyle{title}

Terapia z użyciem jonów o energiach do kilkuset MeV/u jest bardzo obiecująca, nie tylko w przypadku chorób nowotworowych.  Jej rozwój wymaga wielu testów oraz dokładnych obliczeń teoretycznych. Jednym z głównych problemów, stosowanych obecnie modeli,  jest średni potencjał jonizacyjny ośrodka $I$, przez który przechodzi cząstka naładowana. Wpływa on silnie na zasięg cząstek - im wyższa jego wartość tym większy zasięg [1]. Na przykład wartość średniego potencjału wody waha się w przedziale 67,2 eV - 85 eV [2,3], co daje centymetrowy rozrzut zasięgów jonów węgla o energii ??? . Podczas leczenia istotna jest precyzja, dlatego bardzo ważne jest, by w zależności od głębokości umiejscowienia zmiany nowotworowej, dokładnie określić wartość energii jaką powinny mieć cząstki, a to jest możliwe jedynie wtedy, gdy w modelach będzie stosowana "rzeczywista" wartość potencjału $I$. 
Podczas prezentacji przedstawiona zostanie próba estymacji średniego potencjału jonizacyjnego w oparciu o  wyniki literaturowe.

\end{document}
