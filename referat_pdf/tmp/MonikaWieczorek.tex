\documentclass[12pt, a4paper]{article}

%polskie znaki

\usepackage{polski}
\usepackage[utf8]{inputenc}

%matematyczne znaki i strza?eczki
\usepackage{amsmath}
\usepackage{amssymb}
\usepackage{graphicx}
\usepackage{fancyhdr}
\pagestyle{fancy}
\fancypagestyle{title}{%
  %\setlength{\headheight}{22pt}%
  \fancyhf{}% No header/footer
  %\renewcommand{\headrulewidth}{0pt}% No header rule
  %\renewcommand{\footrulewidth}{0pt}% No footer rule
  \fancyfoot[L]{FIZYKA DLA MEDYKA 2016}% Page number in Centre of footer
  \fancyhead[L]{Sesja referatowa}

}%

%nowozdefiniowane symbole
\newcommand{\n}{\mathbb{N}}
\newcommand{\re}{\mathbb{R}}
\newcommand{\limm}{\lim \limits_{x\rightarrow x_0}}
\usepackage{graphicx}


\usepackage[left=1 cm, right=1 cm, top=2 cm, bottom=2 cm]{geometry}

\begin{document}

\title{Brachyterapia nowotworów prostaty - planowanie i realizacja leczenia w czasie rzeczywistym - stare wino w nowej butelce?}
\author{Monika Wieczorek \\ Uniwersytet im. Adama Mickiewicza w Poznaniu}
\date {}
\maketitle
\thispagestyle{title}

Historia brachyterapii sięga początków XIX wieku i jest pierwszą metodą radioterapii. Odkrycie promieniowania X przez H. Becquerela w 1896 roku oraz radu przez małżeństwo Curie w 1898 roku spowodowało istotny progres w rozwoju medycyny. Nowe odkrycia szybko znalazły zastosowanie w leczeniu nowotworów metodą radioterapii. 

Pierwsze aplikacje stosowane były bezpośrednio na zmienione miejsce, a aplikator zbudowany był jedynie z osłony woskowej, w której znajdował się rad 226.  Wraz z upływem lat następował rozwój leczenia nowotworów oraz rozwijały się nowe techniki aplikacji. Do leczenia zaczęto używać nowych źródeł promieniotwórczych, m.in. irydu 192. Momentem przełomowym w leczeniu metodą brachyterapii okazała się realizacja leczenia w czasie rzeczywistym i metoda optymalizacji rozkładów dawek, o których będzie mowa w prezentowanym referacie. Dzięki takiej metodzie w trakcie zabiegu zaistniała możliwość modyfikacji położenia oraz czasów postoju źródeł, co umożliwia indywidualne leczenie pacjenta oraz ochrona narządów krytycznych i tkanek zdrowych.

Celem niniejszego referatu będzie przedstawienie genezy radioterapii, rodzajów źródeł promieniotwórczych oraz metod brachyterapii. Zostanie również podjęta próba udzielenia odpowiedzi na pytanie, będące częścią tytułu prezentacji.

\end{document}
