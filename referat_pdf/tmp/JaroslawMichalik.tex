\documentclass[12pt, a4paper]{article}

%polskie znaki

\usepackage{polski}
\usepackage[utf8]{inputenc}

%matematyczne znaki i strza?eczki
\usepackage{amsmath}
\usepackage{amssymb}
\usepackage{graphicx}
\usepackage{fancyhdr}
\pagestyle{fancy}
\fancypagestyle{title}{%
  %\setlength{\headheight}{22pt}%
  \fancyhf{}% No header/footer
  %\renewcommand{\headrulewidth}{0pt}% No header rule
  %\renewcommand{\footrulewidth}{0pt}% No footer rule
  \fancyfoot[R]{FIZYKA DLA MEDYKA 2016}% Page number in Centre of footer
  \fancyhead[R]{Sesja referatowa}

}%

%nowozdefiniowane symbole
\newcommand{\n}{\mathbb{N}}
\newcommand{\re}{\mathbb{R}}
\newcommand{\limm}{\lim \limits_{x\rightarrow x_0}}
\usepackage{graphicx}


\usepackage[left=1 cm, right=1 cm, top=2 cm, bottom=2 cm]{geometry}

\begin{document}

\title{Dynamika dotyku jako cecha biometryczna }
\author{Jarosław Michalik, Elwira Nowiszewska \\Akademia Górniczo-Hutnicza im. Stanisława Staszica w Krakowie\\Wydział Fizyki i Informatyki Stosowanej\\ SKNFM Kerma}
\date {}
\maketitle
\thispagestyle{title}
Urządzenia mobilne stanowią istotny element współczesnego świata, co stwarza możliwość akwizycji oraz analizy różnorodnych sygnałów biometrycznych. 
Jednym z nich jest dynamika dotyku, która umożliwia ocenę motoryki badanej osoby oraz sprawdzenie jak wpływają na nią różne czynniki - zewnętrzne i wewnętrzne. Celem projektu, jest realizacja narzędzia - aplikacji mobilnej na platformę Android, zdolnej zebrać omówione dane biometryczne oraz dokonać ich analizy. 

Z wykorzystaniem stworzonej aplikacji otrzymano oraz przebadano próbki pochodzące od osób przemęczonych oraz osób w stanie upojenia alkoholowego, a następnie porównano uzyskane rezultaty z wynikami pochodzącymi od grupy kontrolnej - osób wypoczętych i zdrowych. Na podstawie uzyskanych wyników zbudowano wektor cech, starając się znaleźć wielkości charakteryzujące sygnał i opisujące go w sposób jednoznaczny. Celem projektu jest stworzenie narzędzia o praktycznym zastosowaniu - podręcznego miernika zdolności psychomotorycznej, który posłużyć może osobom, których zawodowa praca wymaga koncentracji i precyzyjnej sprawności ruchowej.

\end{document}
