\documentclass[12pt,a4paper]{article}
\usepackage[utf8]{inputenc}
\usepackage{amsmath}
\usepackage{amsfonts}
\usepackage{amssymb}
\usepackage[polish]{babel}
\usepackage[T1]{fontenc}
%\usepackage{multirow}
\usepackage[left=0.75cm,right=0.75cm,top=1cm,bottom=1.5cm]{geometry}
\usepackage{setspace}
\usepackage{float}
\author{Beata Trzpil}
\title{Wielokanałowy system do dwukierunkowej komunikacji między komputerem i komórkami nerwowymi}
\begin{document}
Wielokanałowy system do dwukierunkowej komunikacji między komputerem i komórkami nerwowym
\\ \\
Beata Trzpil
\\
Akademia Górniczo-Hutnicza im.Stanisława Staszica w Krakowie
\\

Współcześnie istniejące i wykorzystywane w eksperymentach systemy wieloelektrodowe wykorzystują tylko w niewielkim stopniu potencjał, jaki w zakresie wyrafinowanej stymulacji elektrycznej oferuje technologia matryc mikroelektrodowych o wysokiej gęstości. Do aktywacji komórek typowo stosuje się impulsy prądowe o nieskomplikowanych kształtach. Wyniki symulacji numerycznych sugerują, że możliwa jest znacząca poprawa selektywności stymulacji elektrycznej poprzez zastosowanie stymulacji wieloelektrodowej lub poprzez optymalizację kształtu impulsu. 
Realizacja takich badań jest jednak obecnie możliwa tylko w bardzo ograniczonym zakresie ze względu na ograniczenia współczesnych wieloelektrodowych systemów do stymulacji i rejestracji aktywności neuronalnej. 
\\
System wieloelektrodowy do jednoczesnej stymulacji i rejestracji aktywności komórek nerwowych opracowany we współpracy WFiIS AGH, Uniwersytetu Kalifornijskiego w Santa Cruz, Uniwersytetu Stanforda oraz Uniwersytetu Strathclyde bazuje na specjalnie opracowanych matrycach wieloelektrodowych o wysokiej rozdzielczości oraz dedykowanej elektronice. Od strony sprzętowej system może realizować dowolnie złożone sekwencje sygnałów stymulacyjnych, jednak współpracujące oprogramowanie narzuca dodatkowe ograniczenia. 
\\
W ramach prezentacji przedstawione zostanie samodzielne wykonane  oprogramowanie rozszerzające istniejącą funkcjonalność systemu. Zapewni ono niespotykaną funkcjonalność  - całkowicie arbitralne definiowanie sygnałów stymulacyjnych na wszystkich elektrodach, obejmujących skomplikowane sekwencje impulsów o różnych kształtach i amplitudach, generowanych z jednoczesną realizowaną sprzętowo redukcją artefaktów stymulacyjnych. 
\\ \\ \\
Hottowy P, Skoczeń A, Gunning DE, Kachiguine S, Mathieson K, Sher A, Wiącek P, Litke AM, Dąbrowski W (2012). Properties and applications of a multichannel integrated circuit for low-artifact, patterned electrical stimulation of neural tissue. Journal of Neural Engineering 9: 066005.
\\ \\LitkeAM, Bezayiff N, Chichilnisky EJ,Cunningham W, Dabrowski W, Grillo A, GrivichM, Gryboś P, Hottowy P, Kachiguine, Kalmar RS, Mathieson K, Petrusca D, Rahman M, Sher A (2004). What does the eye tell the brain?: Development of a system for the large-scale recording of retinal output activity. IEEE Transaction on Nuclear Science 51(4), 1434–1440.

\end{document}
