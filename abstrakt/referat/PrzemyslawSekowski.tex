Terapia z użyciem jonów o energiach do kilkuset MeV/u jest bardzo obiecująca, nie tylko w przypadku chorób nowotworowych.  Jej rozwój wymaga wielu testów oraz dokładnych obliczeń teoretycznych. Jednym z głównych problemów, stosowanych obecnie modeli,  jest średni potencjał jonizacyjny ośrodka $I$, przez który przechodzi cząstka naładowana. Wpływa on silnie na zasięg cząstek - im wyższa jego wartość tym większy zasięg [1]. Na przykład wartość średniego potencjału wody waha się w przedziale 67,2 eV - 85 eV [2,3], co daje centymetrowy rozrzut zasięgów jonów węgla o energii ??? . Podczas leczenia istotna jest precyzja, dlatego bardzo ważne jest, by w zależności od głębokości umiejscowienia zmiany nowotworowej, dokładnie określić wartość energii jaką powinny mieć cząstki, a to jest możliwe jedynie wtedy, gdy w modelach będzie stosowana "rzeczywista" wartość potencjału $I$.\\
Podczas prezentacji przedstawiona zostanie próba estymacji średniego potencjału jonizacyjnego w oparciu o  wyniki literaturowe.\\


[1]. J. Soltani-Nabipour, D. Sardari, GH. Cata-Danil,\textit{ Sensitivity of the Bragg Peak curve to the average ionization potential of the stopping medium}, Rom. Journ. Phys., Vol. 54, Nos. 3–4, P. 321–330, 2009\\
[2]. ICRU Report 73, J. ICRU 5, No. 1, 2005\\
[3]. D. Emfietzoglou, A. Pathak, G. Papamichael, K. Kostarelos, S. Dhamodaran, N. Sathish,
M. Moscovitch, \textit{A study on the electronic stopping of protons in soft biological matter}, Nucl. Instr. and Meth. B 242, 55–60, 2006
